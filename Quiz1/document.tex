\documentclass[10pt,a4paper]{article}
\usepackage[utf8]{inputenc}
\usepackage{amsmath}
\usepackage{amsfonts}
\usepackage{amssymb}
\usepackage{makeidx}
\usepackage{graphicx}
\usepackage{fullpage}
\usepackage{pgffor}

\usepackage{listings}
\usepackage{caption}
\usepackage{subcaption}
\captionsetup[lstlisting]{font=small}
\lstset{
	language=C,                % choose the language of the code
	numbers=left,                   % where to put the line-numbers
	stepnumber=1,                   % the step between two line-numbers.        
	numbersep=5pt,                  % how far the line-numbers are from the code
	showspaces=false,               % show spaces adding particular underscores
	showstringspaces=false,         % underline spaces within strings
	showtabs=false,                 % show tabs within strings adding particular underscores
	tabsize=2,                      % sets default tabsize to 2 spaces
	captionpos=b,                   % sets the caption-position to bottom
	breaklines=true,                % sets automatic line breaking
	breakatwhitespace=true,         % sets if automatic breaks should only happen at whitespace
}

\title{\centering 
	EED1005 Introduction to Programming\\
	Laboratory Quiz 1 \\
	\small \hfill Groups D-E-F}
\begin{document}
	\maketitle	
	\begin{enumerate}
		\item[$Q1$] Draw the flowchart of the C program that prints the result given in Table \ref{tab: table1}.
		
		\begin{table}[!htbp]
		\centering
		\begin{tabular}{c c c c c c c c c}
		01 & 02 & 03 & 04 & 05 & 06 & 07 & 08 & 09 \\
		02 & 04 & 06 & 08 & 10 & 12 & 14 & 16 & 18 \\
		03 & 06 & 09 & 12 & 15 & 18 & 21 & 24 & 27\\
		04 & 08 & 12 & 16 & 20 & 24 & 28 & 32 & 36 \\
		05 & 10 & 15 & 20 & 25 & 30 & 35 & 40 & 45 \\
		06 & 12 & 18 & 24 & 30 & 36 & 42 & 48 & 56 \\
		07 & 14 & 21 & 28 & 35 & 42 & 49 & 56 & 63 \\
		08 & 16 & 24 & 32 & 40 & 48 & 56 & 64 & 72 \\
		09 & 18 & 27 & 36 & 45 & 54 & 63 & 72 & 81 \\
		\end{tabular}
		
		\caption{Output of the C program for $Q1$.}
		\label{tab: table1}
		\end{table}
		
		\newpage
		\item[$Q2$] Fill in the blanks when the C program given in Listing \ref{listing1}  is executed.
		\lstinputlisting[label=listing1, caption=C program for $Q2$.]{main1.c}
		
		
		Output1 = $\_\_\_\_\_\_$\\
		
		Output2 = $\_\_\_\_\_\_$\\
		
		Output3 = $\_\_\_\_\_\_$\\
		
		Output4 = $\_\_\_\_\_\_$\\
		
		Output5 = $\_\_\_\_\_\_$\\
		
		Output6 = $\_\_\_\_\_\_$\\
	\end{enumerate}
\end{document}